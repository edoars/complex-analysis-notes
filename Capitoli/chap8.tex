%!TEX root = ../main.tex
\chapter{Teorema della mappa di Riemann}
%%%%%%%%%%%%%%%%%%%%%%%%%%%%%%%%%%%%%%%%%%
%
%LEZIONE 16/12/2016 - DECIMA SETTIMANA (2)
%
%%%%%%%%%%%%%%%%%%%%%%%%%%%%%%%%%%%%%%%%%%
%%%%%%%%%%%%%%%%%%%%%%%%%%%%%%%%%%%%%%%%%%%%%%%%%%
%COMPATTEZZA NELLO SPAZIO DELLE FUNZIONI OLOMORFE%
%%%%%%%%%%%%%%%%%%%%%%%%%%%%%%%%%%%%%%%%%%%%%%%%%%
\section{Compattezza nello spazio delle funzioni olomorfe}

\begin{defn}{Famiglia di funzioni normale}{famigliaNormale}\index{Famiglia normale}
	Siano \(\Omega\subseteq\C\) aperto e \(F\subseteq H(\Omega)\).
	Diciamo che \(F\) è una \emph{famiglia normale} se per ogni successione di funzioni \(\{f_n\}_{n\ge 1}\subseteq F\), trovo una sottosuccessione \(\{f_{n_h}\}_{h\ge 1}\) e \(f\in H(\Omega)\) tali che
	\[
		f_{n_h} \longrightarrow f \qquad\text{quasi uniformemente}.
	\]
\end{defn}

\begin{oss}
	Non chiedendo che \(f\in F\), non abbiamo trovato una reale definizione di compattezza, ma, piuttosto, una di relativa compattezza.
	Ricordiamo infatti che un insieme si dice relativamente compatto se la sua chiusura è compatta.
\end{oss}

\begin{teor}{di Ascoli-Arzelà}{teoremaAscoliArzelà}
	Sia \(K\subseteq\R^2\) un compatto e sia \(\{f_n\}_{n\ge 1}\subseteq C(K,\R)\) una successione di funzioni tale che
	\begin{enumerate}
		\item \(\displaystyle \sup_{\substack{z\in K\\n\in\N}} \abs{f_n(z)} < +\infty\).
		\item Esiste \(L>0\) tale che
		      \[
			      \abs{f_n(z_1)-f_n(z_2)} \le L\,\abs{z_1-z_2},\,\fa z_1,z_2\in K\,\fa n\in\N.
		      \]
	\end{enumerate}
	Allora esiste una sottosuccessione \(\{f_{n_h}\}_{h\ge 1}\) che converge a una funzione \(f\) uniformemente su \(K\).
\end{teor}

\begin{proof}
	Risultato di Analisi II.
\end{proof}

\begin{oss}
	L'ipotesi \((1)\) da sola non è sufficiente. \(\sin \frac{1}{x^n}\) è un esempio di funzione limitata che non converge a una funzione continua
\end{oss}

\begin{teor}{Caratterizzazione delle famiglie normali}{caratterizzazioneFamiglieNormali}
	Sia \(\Omega\subseteq\C\) un aperto e sia \(F\subseteq H(\Omega)\).
	\(F\) è normale se e soltanto se per ogni \(K\subseteq\Omega\) compatto, esiste \(M>0\) tale che
	\[
		\abs{f_n(z)} \le M,\,\fa n\ge 1\,\fa z\in K.
	\]
\end{teor}

\begin{proof}
	\graffito{\(\Rightarrow)\)}Supponiamo per assurdo che \(F\) non sia equilimitata sui compatti di \(\Omega\). Possiamo quindi trovare \(K\subseteq\Omega\) tale che
	\[
		\sup_{z\in K}\abs{f_n(z)} \ge n.
	\]
	Per ipotesi \(F\) è normale, quindi posso estrarre una sottosuccessione \(\{f_{n_h}\}\) di \(\{f_n\}\) che converge a \(f\) uniformemente sui compatti di \(\Omega\). In particolare
	\[
		\sup_{z\in K} \abs{f_{n_h}(z)-f(z)} \longrightarrow 0 \qquad\text{per }h \to +\infty.
	\]
	Ora, per Weierstrass, \(f\) è limitato su \(K\), quindi esiste \(M>0\) tale che
	\[
		\sup_{z\in K} \abs{f(z)} \le M.
	\]
	Da cui
	\[
		n_h \le \sup_{z\in K}\abs{f_{n_h}(z)} \le \sup_{z\in K} \abs{f_{n_h}(z)-f(z)} + M.
	\]
	che implica
	\[
		\lim_{h \to +\infty} n_h \le M
	\]
	che è assurdo.
	
	\graffito{\(\Leftarrow)\)}Sia \(K\subseteq \Omega\) compatto. Per compattezza posso ricoprire \(K\) con un numero finito di palle
	\[
		\big\{B(z_i;r_i)\big\}_{i=1}^l \qquad\text{con }\chius{B(z_i;r_i)} \subseteq \Omega.
	\]
	Per dimostrare che \(F\) è normale è quindi sufficiente dimostrare che \(\{f_n\}\) è compatto in ogni palla \(B(z_0;r)\) con \(\chius{B(z_0;r)}\subseteq\Omega\).
	Seguirebbe infatti che è possibile estrarre \(\{f_{n_h}\}\) che converge su \(\chius{B(z_1;r_1)}\). Inoltre, dal momento che le \(B(z_i;r_i)\) sono finite, posso iterare:
	da \(\{f_{n_h}\}\) estraggo quindi \(\{f_{n_{h_j}}\}\) che converge anche su \(\chius{B(z_2;r_2)}\) e così fino ad estrarre una sottosuccessione \(\{f_{n_s}\}\) che converge su tutte le palle. In particolare essa convergerebbe su \(K\) in quanto
	\[
		\bigcup_{i=1}^l B(z_i;r_i) \supseteq K.
	\]
	Sia \(B(z_0;r)\) tale che \(\chius{B(z_0;r)}\subseteq \Omega\) e sia \(\{f_n\}\subseteq F\).
	Sfruttiamo Ascoli-Arzelà per dimostrare che \(\{f_n\}\) è compatto su \(B(z_0;r)\).
	Per ipotesi
	\[
		\abs{f_n(z)} \le M,\,\fa n \ge 1\,\fa z\in \chius{B(z_0;r)}
	\]
	che è la prima ipotesi di Ascoli-Arzelà. Grazie al teorema di Lagrange, per mostrare la seconda, è sufficiente verificare che
	\[
		\abs{f_n'(z)} \le M',\,\fa n\ge 1\,\fa z\in \chius{B(z_0;r)}.
	\]
	Sfruttiamo la formula integrale di Cauchy:
	prendiamo \(\e>0\) tale che \(\chius{B(z_0;r+\e)}\subseteq \Omega\). In particolare avremo ancora
	\[
		\abs{f_n(z)} \le M'',\,\fa n\ge 1\,\fa z\in \chius{B(z_0;r+\e)}\subseteq \Omega.
	\]
	Da cui \graffito{abbiamo sfruttato la parametrizzazione \(w=z_0+(r+\e)e^{i\,t}\) e abbiamo osservato che \(\e\) è la minima distanza possibile tra \(w\) e \(z\)}
	\[
		\begin{split}
			\abs{f_n'(z)} & = \bigg\lvert \frac{1}{2\p\,i} \int\limits_{\mathclap{\pd B(z_0;r+\e)}} \frac{f_n(w)}{(w-z)^2}\,\dd w \bigg\rvert = \bigg\lvert \frac{1}{2\p\,i} \int_0^{2\p} \frac{f_n\big(z_0+(r+\e)e^{i\,t}\big)}{\big[\big(z_0+(r+\e)e^{i\,t}\big)-z\big]^2}\,(r+\e)i\,e^{i\,t}\,\dd t\bigg\rvert\\
			& \le \frac{1}{2\p} \int_0^{2\p} \frac{M''}{\e^2}\,(r+\e)\,\dd t = \frac{M''}{\e^2}\,(r+\e)\,\fa z\in \chius{B(z_0;r)}.\qedhere
		\end{split}
	\]
\end{proof}
%%%%%%%%%%%%%%%%%%%%%%%%%%%%%%%%%%%%%%%%%%%%%%
%
%LEZIONE 20/12/2016 - UNDICESIMA SETTIMANA (1)
%
%%%%%%%%%%%%%%%%%%%%%%%%%%%%%%%%%%%%%%%%%%%%%%
%%%%%%%%%%%%%%%%%%
%MAPPA DI RIEMANN%
%%%%%%%%%%%%%%%%%%
\section{Mappa di Riemann}

\begin{teor}{della mappa di Riemann}{teoremaMappaRiemann}\index{Teorema!della mappa di Riemann}
	Sia \(\Omega\subset\C\) un aperto semplicemente connesso che non coincida con \(\C\).
	Allora esiste \(f\colon \Omega \to B(0;1)\) biiettiva e biolomorfa.
	Inoltre, preso \(z_0\in\Omega\), si può imporre
	\begin{align*}
		f(z_0) & = 0, & f'(z_0) & \in \R, & f'(z_0) & > 0.
	\end{align*}
	\(f\), con questa condizione, è unica.
\end{teor}

\begin{proof}
	Mostriamo l'esistenza di \(f\). Consideriamo
	\[
		F = \Set{f\colon \Omega \to B(0;1) | f\text{ olomorfa, iniettiva e }f(z_0)=0}.
	\]
	Vogliamo dimostrare che
	\begin{enumerate}
		\item \(F\) non è vuoto.
		\item \(\sup\Set{\abs{f'(z_0)}:f\in F}\) è un massimo.
		\item Se \(f_0\in F\) è massimale rispetto a \((2)\), allora \(f_0\) è la funzione cercata.
	\end{enumerate}
	\graffito{\(1)\)}Per ipotesi \(\Omega\) è semplicemente connesso e \(\Omega\neq \C\), quindi esiste \(w_0\not\in\Omega\).
	Per cui la mappa \(w\mapsto w-w_0\) a valori in \(\Omega\) non si annulla mai.
	Per il \hyperref[pr:determinazioneRadice]{teorema di monodromia} possiamo definire
	\[
		\j\colon \Omega \longrightarrow \C, z \longmapsto \sqrt{z-w_0}.
	\]
	\(\j\) è iniettiva, infatti
	\[
		\j(z_1) = \j(z_2) \implies z_1-w_0=z_2-w_0 \implies z_1 = z_2.
	\]
	Inoltre  \(\j(z_1)\neq -\j(z_2)\) se \(z_1\neq z_2\), infatti
	\[
		\j(z_1) = -\j(z_2) \implies \j(z_1)^2=\j(z_2)^2 \implies z_1-w_0 = z_2-w_0 \implies z_1 = z_2.
	\]
	Ciò significa che se \(\j\) assume il valore \(c\), allora non assume il valore \(-c\).
	Sappiamo che le \hyperref[th:funzioniOlomorfeAperte]{mappe olomorfe sono aperte}, quindi \(\j(\Omega)\) è aperto.
	Ora se \(B(a;r)\subseteq \j(\Omega)\), per quanto appena detto, avremo
	\[
		-B(a;r) = B(-a;r) \subseteq \setc{\j(\Omega)}.
	\]
	Definiamo
	\[
		\y\colon \Omega \longrightarrow B(0;1), z \longmapsto \frac{r}{\j(z)+a}.
	\]
	Osserviamo che \(\y\) è iniettiva in quanto \(\j\) lo è.
	Inoltre \(\y\) è olomorfa poiché è l'inverso di una funzione olomorfa che non si annulla mai, infatti abbiamo dimostrato che \(\j(z)\not\in B(-a;r)\) per ogni \(z\in\Omega\).
	Infine
	\[
		\j(z) \not\in B(-a;r) \implies \abs{\j(z)+a} > r \implies \y(\Omega) \subseteq B(0;1).
	\]
	Quindi, componendo \(\y\) con un automorfismo del disco che manda \(\j(z_0)\) in \(0\), ho trovato un elemento di \(F\).
	
	\graffito{\(2)\)}Sia \(\a = \sup\Set{\abs{f'(z_0)}:f\in F}\). Se \(f\in F\) per definizione \(\abs{f(z)}\le 1\) per ogni \(z\in \Omega\), quindi \(F\) è una famiglia normale per la \hyperref[th:caratterizzazioneFamiglieNormali]{caratterizzazione di queste ultime}.
	Dalla definizione di estremo superiore, trovo una successione massimizzante, ovvero
	\[
		\{f_n\}\subseteq F \qquad\text{tale che }\abs{f_n'(z_0)} \ge \a - \frac{1}{n}.
	\]
	D'altronde \(F\) è normale, quindi esiste una sottosuccessione \(\{f_{n_h}\}\) che converge quasi uniformemente a \(f_0\) su \(\Omega\). Vogliamo mostrare che
	\[
		f_0 \in F \qquad\text{e}\qquad \abs{f_0'(z_0)} = \a.
	\]
	\'E chiaro che \(f_0(\Omega)\subseteq B(0;1)\), inoltre \(f_0(z_0)=0\) perché
	\[
		0 = f_n(z_0) \longrightarrow f(z_0).
	\]
	Infine \(f_0\) è iniettivo perché, \hyperref[pr:successioneFunzioniIniettive]{da un risultato precedente}, il limite quasi uniforme di una successione di funzioni iniettive è iniettivo o costante; d'altronde \(f_0\) non è costante perché, come dimostreremo tra poco, \(\abs{f_0'(z_0)} = \a\) e quindi
	\[
		\abs{f_0'(z_0)} > \abs{\tilde{f}'(z_0)} \implies f_0 \not\equiv 0
	\]
	\(\abs{f_0'(z_0)}=\a\) perché se \(f_n\) converge quasi uniformemente a \(f_0\),  come conseguenza della formula di Cauchy, anche \(f_n'\) converge quasi uniformemente a \(f_0'\).
	
	\graffito{\(3)\)}\'E sufficiente mostrare che \(f_0\) è suriettiva, così da ottenere la funzione biiettiva e olomorfa, e di conseguenza biolomorfa, cercata.
	Supponiamo per assurdo che \(f_0\) non sia suriettiva, mostriamo che questo ci permette di costruire una mappa \(h\in F\) tale che \(\abs{h'(z_0)}>\a\) contro la massimalità di \(\a\).
	Indichiamo con \(\j_\b\) la generica mappa di M\"oebius
	\[
		\j_\b\colon B(0;1) \longrightarrow B(0;1), z \longmapsto \frac{z-\b}{1-\conj{\b}z}.
	\]
	Sia \(\g\in B(0;1)\) tale che \(\g\not\in f_0(\Omega)\), componiamo \(f_0\) con \(\j_\g\) così da avere
	\[
		0\not\in \j_\g\circ f_0(\Omega).
	\]
	Dal \hyperref[pr:determinazioneRadice]{teorema di monodromia} sugli aperti semplicemente connessi che escludono l'origine, si riesce a definire un ramo della radice. Sia quindi \(s(z)=z^2\) e componiamo \(\j_\g\circ f_0\) con \(s^{-1}\) chiamando la funzione ottenuta \(g\).
	Componiamo infine con \(\j_{g(z_0)}\), abbiamo quindi
	\[
		h = \j_{g(z_0)} \circ s^{-1} \circ \j_\g \circ f_0.
	\]
	Ora \(h\in F\) come immediata conseguenza della sua definizione. Mostriamo che \(\abs{h'(z_0)}>\a\).
	Per costruzione
	\[
		h(z) = \j_{g(z_0)} \circ s^{-1} \circ \j_\g \circ f_0(z) \implies f_0(z) = \underbracket{\j_{\g}^{-1}\circ s\circ \j_{g(z_0)}^{-1}}_{G} \circ h(z),
	\]
	da cui, dal momento che \(h(z_0)= 0\),
	\[
		0 = f_0(z_0) = G\circ h(z_0) \implies F(0) = 0.
	\]
	Inoltre \(G\) porta \(B(0;1)\) in sé, quindi, per il \hyperref[th:lemmaSchwarz]{lemma di Schwarz}, \(\abs{G'(0)}\le 1\). D'altronde \(G\) non è una rotazione, quindi \(\abs{G'(0)}\neq 1\). Infine, per la regola della catena,
	\[
		\abs{f_0'(z_0)} = \abs{G'(0)}\abs{h'(z_0)} \implies \abs{h'(z_0)} > \abs{f_0'(z_0)},
	\]
	che è assurdo.
	
	Mostriamo l'unicità. Siano \(f,g\colon \Omega \to B(0;1)\) due mappe biiettive e biolomorfe che soddisfano la condizione di \(z_0\). Dimostriamo che \(f\equiv g\).
	Sappiamo che tali mappe sono invertibili. Consideriamo \(g^{-1}\) e definiamo
	\[
		h(z)\colon B(0;1) \longrightarrow B(0;1), z \longmapsto f\circ g^{-1}(z).
	\]
	\(h\) è ancora biiettiva e biolomorfa, quindi \(h\in \Aut\big(B(0;1)\big)\).
	Inoltre \(h(0)=0\), quindi dalla caratterizzazione degli automorfismi del disco e dal lemma di Schwarz, segue
	\[
		h(z) = e^{i\,\q}z.
	\]
	Ora \(g',f'\in \R^+\), da cui
	\[
		e^{i\,\q} = h'(0) = f'|_{g^{-1}(0)}(g^{-1})'(0) = \frac{f'(z_0)}{g'(z_0)} \in \R^+ \implies \q = 0 \implies h(z) = z.
	\]
	Quindi
	\[
		f\circ g^{-1}(z) = z \iff f(z) = g(z),\,\fa z\in \Omega.
	\]
\end{proof}

\begin{oss}
	\'E sempre possibile imporre la condizione di unicità. Infatti trovata \(f\) è sufficiente comporla prima con l'automorfismo del disco che manda \(f(z_0)\) in \(0\):
	\[
		z \longmapsto \frac{z-f(z_0)}{1-\conj{f(z_0)}z};
	\]
	e infine comporre con la rotazione
	\[
		z \longmapsto z\,\frac{\conj{g'(z_0)}}{\abs{g'(z_0)}},
	\]
	dove \(g\) è \(f\) composto con l'automorfismo.
\end{oss}

\begin{oss}
	L'ipotesi di \(\Omega \neq \C\) è necessaria in quanto, per il \hyperref[th:teoremaLiouville]{teorema di Liouville}, una mappa olomorfa \(f\colon \C \to B(0;1)\) è costante.
\end{oss}

\begin{teor}{di Picard}{teoremaPicard}\index{Teorema!di Picard}
	Sia \(f\colon \C \to \C\) una funzione intera.
	Se \(f\) evita almeno due valori, cioè
	\[
		f(\C) \subseteq \C\setminus\{z_1,z_2\},
	\]
	allora \(f\) è costante.
\end{teor}

\begin{proof}
	Sia \(f\) una funzione intera che evita \(z_1,z_2\). Non è restrittivo pensare \(z_1=0\) e \(z_2=1\), è infatti sufficiente applicare una rotodilatazione adeguata.
	
	Se dimostriamo che \(B(0;1)\) è il ricoprimento universale di \(\C\setminus\{0,1\}\), per un teorema di Topologia sui sollevamenti, troviamo una mappa olomorfa \(g\) che fa commutare il seguente diagramma
	\[
		\begin{tikzcd}
			& B(0;1) \arrow{d}{\p}\\
			\C \arrow{ru}{g} \arrow[swap]{r}{f} & \C\setminus\{0,1\}
		\end{tikzcd}
	\]
	Per \hyperref[th:teoremaLiouville]{Liouville} \(g\) risulterà costante e pertanto lo sarebbe anche \(f=\p\circ g\).
	
	Mostriamo che \(B(0;1)\) copre \(\C\setminus\{0,1\}\). Chiamo \(S^{\pm}\) i semipiani inferiore e superiore.
	Prendiamo un triangolo equilatero su \(B(0;1)\) e denotiamo l'insieme dei punti che contiene con \(D_0\).
	Per la mappa di Riemann, trovo
	\[
		\j\colon D_0 \longrightarrow S^+
	\]
	continua fino al bordo e tale che
	\begin{align*}
		\j(1) & = 0, & \j(e^{i\,\frac{2\p}{3}}) & = 1, & \j(e^{i\,\frac{4\p}{3}}) & = \infty.
	\end{align*}
	Siano \(a,b,c\) i tre archi che delimitano \(D_0\).
	Sia \(R\) l'inversione circolare rispetto ad \(a\) in \(B(0;1)\). Tramite \(R\) la circonferenza unitaria è portata in una circonferenza ortogonale ad \(a\) che passa per \(1\) e \(e^{i\,\frac{2\p}{3}}\), che è pertanto ancora \(B(0;1)\).
	Su \(R(D_0)\) definiamo
	\[
		\tilde{\j}\colon R(D_0) \longrightarrow S^-, z \longmapsto \conj{\j}(Rz).
	\]
	Per il \hyperref[th:principioRiflessioneSchwarz]{principio di riflessione di Schwarz} la mappa
	\[
		\tilde{\j}\colon D_0\cup R(D_0) \longrightarrow \C
	\]
	è olomorfa.
	Iterando per tutti i lati infinite volte si ottiene il ricoprimento.
\end{proof}

\begin{notz}
	Spesso si fa riferimento a questo teorema come al \emph{piccolo teorema di Picard}.
\end{notz}

\begin{oss}
	Il teorema non può essere rifinito ulteriormente, vi sono infatti funzioni intere che evitano un solo valore ma non sono costanti. Ad esempio \(e^z\) ha immagine \(\C\setminus\{0\}\).
\end{oss}