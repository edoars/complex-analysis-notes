%!TEX root = ../main.tex
\chapter{Calcolo di integrali definiti}
%%%%%%%%%%%%%%%%%%%%%%%%%%%%%%%%%%%%%%%%%%
%
%LEZIONE 28/10/2016 - QUINTA SETTIMANA (2)
%
%%%%%%%%%%%%%%%%%%%%%%%%%%%%%%%%%%%%%%%%%%
%%%%%%%%%%%%%%%%%%%%%%%%%%%%%%%%
%ESEMPI SUL TEOREMA DEI RESIDUI%
%%%%%%%%%%%%%%%%%%%%%%%%%%%%%%%%
\section{Esempi sul teorema dei residui}
%\setcounter{exeN}{0}

In questi esempi calcoleremo alcuni integrali di Riemann, che normalmente risulterebbero complicati, tramite l'applicazione del \hyperref[th:teoremaResiduiGeneralizzato]{teorema dei residui generalizzato}.

Per prima cosa analizziamo un caso standard, in cui l'integrando consiste in un polinomio razionale.

\begin{prop}{Integrale su \(\R\) di un polinomio razionale}{integraleRPolinomioRazionale}
	Siano \(P(x),Q(x)\) polinomi a coefficienti reali.
	Supponiamo che \(Q(x)\) non abbia zeri reali e che \(\deg Q \ge \deg P+2\), allora
	\[
		\int_{-\infty}^{+\infty} \frac{P(x)}{Q(x)}\,\dd x = 2\p\,i \sum_{i} \Set{\Res \left( \frac{P(z)}{Q(z)},z_i \right) | z_i \text{ polo del semipiano superiore}}.
	\]
\end{prop}

\begin{proof}
	Le ipotesi su \(Q(x)\) e \(P(x)\) sono necessarie affinché l'integrale converga secondo Riemann.
	\[
		\tikz[baseline=-0.5ex]{
	\draw [-latex, thick] (-2,0) -- (2.3,0);
	\draw [-latex, thick] (0,-1) -- (0,2.3);
	\draw [
		decoration={markings, mark=at position 0.25 with {\arrow{>}}},
		postaction={decorate}
		] (1.7,0) arc (0:180:1.7);
	\draw [
		decoration={markings, mark=at position 0.75 with {\arrow{>}}},
		postaction={decorate}
		] (-1.7,0) -- (1.7,0);

	\node [above left, font=\large] at (2.3,0) {\(x\)};
	\node [below right, font=\large] at (0,2.3) {\(i\,y\)};
	\node [below, font=\footnotesize] at (1.7,0) {\(R\)};
	\node [above, font=\footnotesize] at (0.85,0.05) {\(\g_1^R\)};
	\node [above right, font=\footnotesize] at (45:1.7) {\(\g_2^R\)};
}
	\]
	I poli della funzione \(f(z)=P(z)/Q(z)\) sono gli zeri complessi del polinomio \(Q(x)\).
	Consisderiamo il seguente ciclo, come mostrato in figura:
	\[
		\g_1^R\colon [-R,R] \to \C, t \mapsto t \qquad\text{e}\qquad \g_2^R\colon [0,\p] \to \C, t \mapsto R\,e^{i\,t}.
	\]
	Osserviamo che per \(R\) sufficientemente grande il ciclo contiente tutti e soli i poli di \(f\) che si trovano nel semipiano superiore.
	Inoltre se \(z_i\) è un polo di \(f\) nel semipiano superiore avremo, per costruzione
	\[
		\ind(\g_1^R+\g_2^R,z_i) = 1.
	\]
	Quindi applicando il \hyperref[th:teoremaResiduiGeneralizzato]{teorema dei residui generalizzato} avremo
	\[
		\int\limits_{\mathclap{\g_1^R+\g_2^R}} \,\frac{P(z)}{Q(z)}\,\dd z = 2\p\,i \sum_i \Res(f,z_i).
	\]
	Andiamo quindi a calcolare l'integrale:
	\[
		\int\limits_{\mathclap{\g_1^R+\g_2^R}} \,\frac{P(z)}{Q(z)}\,\dd z = \int\limits_{\g_1^R}\frac{P(z)}{Q(z)}\,\dd z + \int\limits_{\g_2^R} \frac{P(z)}{Q(z)}\,\dd z = \int_{-R}^R \frac{P(t)}{Q(t)}\,\dd t + \int_0^\p \frac{P(R\,e^{i\,t})}{Q(R\,e^{i\,t})}\,i\,R\,e^{i\,t}\,\dd t.
	\]
	Dove
	\[
		\int_{-R}^R \frac{P(t)}{Q(t)}\,\dd t \xrightarrow{R\to +\infty} \int_{-\infty}^{+\infty} \frac{P(t)}{Q(t)}\,\dd t
	\]
	che è proprio l'integrale di partenza. Mentre se per il secondo ne consideriamo il modulo, avremo
	\[
		\begin{split}
			\bigg\lvert\int_0^\p \frac{P(R\,e^{i\,t})}{Q(R\,e^{i\,t})}\,i\,R\,e^{i\,t}\,\dd t\bigg\rvert & \le \int_0^\p \bigg\lvert \frac{P(R\,e^{i\,t})}{Q(R\,e^{i\,t})}\bigg\rvert\,R\,\dd t \le \int_0^\p \frac{C}{R^2}\,R\,\dd t\graffito{ricordiamo che \(\deg Q\ge\deg P+2\) quindi per \(R\) grande posso stimare il modulo come \(C/R^2\)}\\
			& = \frac{C}{R}\,\p \xrightarrow{R\to +\infty} 0.
		\end{split}
	\]
	Per cui
	\[
		\int_{-\infty}^{+\infty} \frac{P(x)}{Q(x)}\,\dd x = 2\p\,i \sum_i \Res(f,z_i).\qedhere
	\]
\end{proof}

\begin{exeN}
	Calcolare il seguente integrale
	\[
		\int\limits_\R \frac{x^2}{1+x^4}\,\dd x.
	\]
\end{exeN}

\begin{sol}
	Osserviamo che il denominatore non ha radici reali e che la differenza dei gradi è \(2\).
	Possiamo quindi applicare la proposizione precedente:
	\[
		\int\limits_\R \frac{x^2}{1+x^4}\,\dd x = 2\p\,i \sum_i \Res \left( \frac{z^2}{1+z^4},z_i \right),\text{ dove \(z_i\) è un polo del semipiano superiore}.
	\]
	Calcoliamo quindi i poli della funzione. Osserviamo che gli unici poli sono gli zeri complessi del denominatore:
	\[
		z_1 = e^{i\,\frac{\p}{4}}, \qquad z_2 = e^{i\,\frac{3}{4}\p}, \qquad z_3 = e^{i\,\frac{5}{4}\p}, \qquad z_4 = e^{i\,\frac{7}{4}\p}.
	\]
	Noi siamo interessati solo ai poli del semipiano superiore, quindi nel nostro caso a \(z_1\) e \(z_2\):
	\[
		f(z) = \frac{z^2}{\left( z-e^{i\,\frac{\p}{4}} \right)\left( z-e^{i\,\frac{3}{4}\p} \right)\left( z-e^{i\,\frac{5}{4}\p} \right)\left( z-e^{i\,\frac{7}{4}\p} \right)} = \frac{1}{z-e^{i\,\frac{\p}{4}}} g(z),
	\]
	dove \(g(z)\) è una funzione olomorfa in \(D(e^{i\,\frac{\p}{4}},\e)\), da cui
	\[
		f(z) = \frac{1}{z-e^{i\,\frac{\p}{4}}} \big[g_0+g_1 (z-e^{i\,\frac{\p}{4}})+g_2 (z-e^{i\,\frac{\p}{4}})^2+\ldots\big]
	\]
	Quindi
	\[
		\begin{split}
			\Res \left( f(z),e^{i\,\frac{\p}{4}} \right) & = g_0 = g (e^{i\,\frac{\p}{4}}) = \frac{\left(e^{i\,\frac{\p}{4}}\right)^2}{\left( e^{i\,\frac{\p}{4}}-e^{i\,\frac{3}{4}\p} \right)\left( e^{i\,\frac{\p}{4}}-e^{i\,\frac{5}{4}\p} \right)\left( e^{i\,\frac{\p}{4}}-e^{i\,\frac{7}{4}\p} \right)}\\
			& = \frac{\left( e^{i\,\frac{\p}{4}} \right)^2}{\left( e^{i\,\frac{\p}{4}} \right)^3 (1-i)(1+1)(1+i)} = \frac{1}{4e^{i\,\frac{\p}{4}}}.
		\end{split}
	\]
	Analogamente si trova
	\[
		\Res \left( f(z),e^{i\,\frac{3}{4}\p} \right) = \frac{1}{4e^{i\,\frac{3}{4}\p}}.
	\]
	Da cui
	\[
		\int\limits_\R \frac{x^2}{1+x^4}\,\dd x = 2\p\,i \left( \frac{1}{4e^{i\,\frac{\p}{4}}} + \frac{1}{4e^{i\,\frac{3}{4}\p}} \right) = \frac{\p\,i}{2} \left( e^{-i\,\frac{\p}{4}}-e^{-i\,\frac{3}{4}\p} \right) = \frac{\p}{\sqrt{2}}.
	\]
\end{sol}

\begin{exeN}
	Calcolare il seguente integrale
	\[
		\int\limits_\R \frac{\cos(a\,x)}{1+x^2}\,\dd x, a>0.
	\]
\end{exeN}

\begin{sol}
	Osserviamo che anche in questo caso il denominatore non ha zeri reali e la differenza dei gradi dei polinomi è \(2\).
	Sebbene la proposizione non sia direttamente applicabile, possiamo procedere con una strategia analoga. Osserviamo che
	\[
		\int\limits_\R \frac{\cos(a\,x)}{1+x^2}\,\dd x = \Re \int\limits_\R \frac{e^{i\,a\,x}}{1+x^2}\,\dd x,
	\]
	mostreremo che
	\[
		\int\limits_\R \frac{\cos(a\,x)}{1+x^2}\,\dd x = 2\p\,i \sum_i \Res \left( \frac{e^{i\,a\,z}}{1+z^2},z_i \right), \text{ dove \(z_i\) sono i poli superiori della funzione}.
	\]
	Utilizziamo lo stesso cammino \(\g_1^R+\g_2^R\) della proposizione e applichiamo il teorema dei residui:
	\[
		\int\limits_{\mathclap{\g_1^R+\g_2^R}}\, \frac{e^{i\,a\,z}}{1+z^2}\,\dd z = 2\p\,i \sum_{i} \Res \left( \frac{e^{i\,a\,z}}{1+z^2},z_i \right),
	\]
	dove
	\[
		\int\limits_{\mathclap{\g_1^R+\g_2^R}}\, \frac{e^{i\,a\,z}}{1+z^2}\,\dd z = \int\limits_{\g_1^R} \frac{e^{i\,a\,z}}{1+z^2}\,\dd z + \int\limits_{\g_2^R} \frac{e^{i\,a\,z}}{1+z^2}\,\dd z = \int_{-R}^R \frac{e^{i\,a\,t}}{1+t^2}\,\dd t + \int_0^\p \frac{e^{i\,a\,R\,(\cos t+i\,\sin t)}}{1+R^2 e^{2i\,t}}i\,R\,e^{i\,t}\,\dd t
	\]
	Passando al limite per \(R\to +\infty\), avremo
	\[
		\int_{-R}^R \frac{e^{i\,a\,t}}{1+t^2}\,\dd t \longrightarrow \int_{-\infty}^{+\infty} \frac{e^{i\,a\,t}}{1+t^2}\,\dd t
	\]
	che è l'integrale che vogliamo calcolare, mentre
	\[
		\int_0^\p \frac{e^{i\,a\,R\,(\cos t+i\,\sin t)}}{1+R^2 e^{2i\,t}}i\,R\,e^{i\,t}\,\dd t \longrightarrow 0,
	\]
	poiché
	\[
		\begin{split}
			\bigg\lvert \int_0^\p \frac{e^{i\,a\,R\,(\cos t+i\,\sin t)}}{1+R^2 e^{2i\,t}}\,i\,R\,e^{i\,t}\,\dd t \bigg\rvert & \le \int_0^\p \frac{\big\lvert e^{i\,a\,R\,\cos t}e^{-a\,R\,\sin t}\big\rvert}{R^2-1}\,R\,\dd t = \int_0^\p \frac{e^{-a\,R\,\sin t}}{R^2-1}\,R\,\dd t\\
			& \le \int_0^\p \frac{1}{R^2-1}\,R\,\dd t = \frac{R}{R^2-1}\,\p \xrightarrow{R\to+\infty} 0.
		\end{split}
	\]
	Restano da calcolare i residui.
	D'altronde questi sono in corrispondenza delle radici complesse di \(z^2+1\), che sono \(\pm i\). Poiché a noi interessano solo i poli del semipiano superiore andiamo a calcolare il residuo in \(i\):
	\[
		\frac{e^{i\,a\,z}}{z^2+1} = \frac{e^{i\,a\,z}}{(z-i)(z+i)} = \frac{1}{z-i} \frac{e^{i\,a\,z}}{z+i} = \frac{1}{z-i}\,g(z),
	\]
	dove \(g(z)\) è una funzione olomorfa in \(D(i,\e)\), per cui
	\[
		\frac{e^{i\,a\,z}}{z^2+1} = \frac{1}{z-i} \big[g_0+g_1(z-i)+g_2(z-i)^2+\ldots\big].
	\]
	Quindi
	\[
		\Res \left( \frac{e^{i\,a\,z}}{z^2+1},i \right) = g_0 = g(i) = \frac{e^{-a}}{2i}.
	\]
	Infine
	\[
		\int\limits_\R \frac{\cos(a\,x)}{1+x^2}\,\dd x = \Re \left( 2\p\,i\,\frac{e^{-a}}{2i} \right) = \p\,e^{-a}.
	\]
	Osserviamo che l'uso della parte reale non è stato veramente necessario, questo poiché in
	\[
		\int\limits_\R \frac{e^{i\,a\,x}}{1+x^2}\,\dd x,
	\]
	l'integrale della parte immaginaria si annullava per via della disparità del seno.
\end{sol}

\begin{exeN}
	Calcolare il seguente integrale
	\[
		\int_0^{+\infty} \frac{\dd x}{1+x^n},
	\]
	al variare di \(n\in\N\).
\end{exeN}

\begin{sol}
	Quando \(n\) è pari possiamo scrivere
	\[
		\int_0^{+\infty} \frac{\dd x}{1+x^n} = \frac{1}{2} \int_{-\infty}^{+\infty} \frac{\dd x}{1+x^n}
	\]
	che possiamo risolvere con i metodi precedenti.

	Analizziamo quindi i casi con \(n\) dispari. I poli della funzione sono gli zeri del denominatore, che corrispondono alle radici \(n\)-esime dell'unità.

	Per applicare il teorema dei residui, scegliamo il seguente cammino, che contiene solo il primo polo:
	\[
		\tikz[baseline=-0.5ex, x=2cm, y=2cm]{
	\draw [-latex, thick] (-1.5,0) -- (2,0);
	\draw [-latex, thick] (0,-1.5) -- (0,2);

	\foreach \k in {0,1,2,3,4,5,6,7,8,9} {
		\fill (180/10+360*\k/10:1) circle (0.02);
	}

	\draw [
		decoration={markings, mark=at position 0.75 with {\arrow{>}}},
		postaction={decorate}
		] (0,0) -- (1.5,0);
	\draw [
		decoration={markings, mark=at position 0.5 with {\arrow{>}}},
		postaction={decorate}
		] (1.5,0) arc (0:2*180/10:1.5);
	\draw [
		decoration={markings, mark=at position 0.75 with {\arrow{>}}},
		postaction={decorate}
		] (0,0) -- (2*180/10:1.5);

	\draw [very thin] (0,0) -- (180/10:1);
	\draw [very thin] (0,0) -- (3*180/10:1);

	\draw [help lines] (0.5,0) arc (0:2*180/10:0.5);

	\node [above left, font=\large] at (2,0) {\(x\)};
	\node [below right, font=\large] at (0,2) {\(i\,y\)};
	\node [right, font=\footnotesize] at (180/10:1) {\(e^{i\,\frac{\p}{n}}\)};
	\node [left, font=\footnotesize] at (3*180/10:1) {\(e^{i\,\frac{3\p}{n}}\)};
	\node [below] at (1.5,0) {\(R\)};
	\node [below] at (1.125,0) {\(\g_1^R\)};
	\node [above=0.1cm] at (2*180/10:1.125) {\(\g_3^R\)};
	\node [right] at (180/10:1.5) {\(\g_2^R\)};
	\node [right, font=\scriptsize]  at (180/15:0.5) {\(\frac{2\p}{n}\)};
}
	\]
	dove
	\[
		\g_1^R\colon [0,R] \to \C, t\mapsto t; \qquad \g_2^R\colon [0,\frac{2\p}{n}] \to \C, t\mapsto R\,e^{i\,t}; \qquad \g_3^R\colon [0,R] \to \C, t \mapsto t\,e^{i\,\frac{2\p}{n}}.
	\]
	Per il teorema dei residui avremo
	\[
		2\p\,i\,\Res(f,e^{i\,\frac{\p}{n}}) = \int\limits_{\mathclap{\g_1^R+\g_2^R-\g_3^R}}\, f(z)\,\dd z,
	\]
	dove
	\[
		\int\limits_{\mathclap{\g_1^R+\g_2^R-\g_3^R}}\, f(z)\,\dd z = \int\limits_{\g_1^R}f(z)\,\dd z + \int\limits_{\g_2^R}f(z)\,\dd z - \int\limits_{\g_3^R}f(z)\,\dd z \xrightarrow{R\to+\infty} \left( 1-e^{i\,\frac{2\p}{n}} \right) \int_0^{+\infty} \frac{\dd t}{1+t^n}.
	\]
	Infatti
	\[
		\int\limits_{\g_1^R} f(z)\,\dd z = \int_0^R \frac{\dd t}{1+t^n} \longrightarrow \int_0^{+\infty} \frac{\dd t}{1+t^n}
	\]
	e
	\[
		\int\limits_{\g_3^R} f(z)\,\dd z = \int_0^R \frac{e^{i\,\frac{2\p}{n}}}{1+t^n e^{i\,2\p}}\,\dd t = e^{i\,\frac{2\p}{n}} \int_0^R \frac{1}{1+t^n}\,\dd t.
	\]
	Resta da mostrare che l'integrale di \(\g_2^R\) tende a zero, per farlo studiamone il modulo:
	\[
		\bigg\lvert \int\limits_{\g_2^R} f(z)\,\dd z \bigg\rvert = \bigg\lvert \int_0^{\frac{2\p}{n}} \frac{i\,R\,e^{i\,t}}{1+R^n e^{ni\,t}}\,\dd t \bigg\rvert \le \int_0^{\frac{2\p}{n}} \frac{R}{R^n-1}\,\dd t = \frac{2\p}{n} \frac{R}{R^n-1} \longrightarrow 0.
	\]
	Calcoliamo il residuo. Per farlo utilizzeremo un metodo differente da quello visto in precedente, che sfrutta la regola di de l'H\^opital
	\[
		f(z) = \frac{g_0}{z-e^{i\,\frac{\p}{n}}} + g_1 + g_2 \left( z-e^{i\,\frac{\p}{n}} \right) + \ldots
	\]
	da cui
	\[
		\begin{split}
			\Res \left( f, e^{i\,\frac{\p}{n}} \right) & = g_0 = \lim_{z \to e^{i\,\frac{\p}{n}}} \left( z-e^{i\,\frac{\p}{n}} \right) f(z) = \lim_{z \to e^{i\,\frac{\p}{n}}} \frac{z-e^{i\,\frac{\p}{n}}}{1+z^n}\\
			& \overset{H}{=} \lim_{z \to e^{i\,\frac{\p}{n}}} \frac{1}{n\,z^{n-1}} = \frac{1}{n\,e^{i\,\p\,\frac{n-1}{n}}}.
		\end{split}
	\]
\end{sol}
%%%%%%%%%%%%%%%%%%%%%%%%%%%
%ESEMPI CON DETERMINAZIONE%
%%%%%%%%%%%%%%%%%%%%%%%%%%%
\section{Esempi con determinazione}

Quando si considerano funzioni del tipo \(z^\a\) con \(\a\in \R\) si ha
\[
	z^\a = e^{\a\,\ln z}.
\]
Quindi, per definizione, \(z^\a\) non è univocamente determinato, ma dipende dalla scelta della \emph{determinazione} di \(\ln z\).

Nei prossimi esempi vedremo come ovviare a questo problema, quando questo genere di funzioni si presenta negli integrali.

\begin{exeN}
	Calcolare il seguente integrale
	\[
		\int_0^{+\infty} x^\a \frac{P(x)}{Q(x)}\,\dd x,
	\]
	supponendo che \(\a\in (0,1)\), \(Q\) non abbia zeri reali e \(\deg Q \ge \deg P+2\).
\end{exeN}

\begin{sol}
	Osserviamo che
	\[
		z^\a \frac{P(z)}{Q(z)} = e^{\a\,\ln z} \frac{P(z)}{Q(z)} = e^{\a\,\ln z} R(z).
	\]
	Definiamo \(\ln z\) in \(\C\) meno l'asse reale positivo. Scegliamo inoltre la determinazione in cui \(\ln z \in \R\) se tende all'asse reale positivo da valori nel semipiano superiore e \(\ln z\in \R + 2\p\,i\) se vi tende dal semipiano inferiore.

	Per applicare il teorema dei residui scegliamo il cammino rappresentato in figura, detto anche buco della serratura
	\[
		\tikz[baseline=-0.5ex]{
	\draw [-latex, thick] (-2.2,0) -- (2.5,0);
	\draw [-latex, thick] (0,-2.2) -- (0,2.5);

	\coordinate (A) at (7.18076:2);
	\coordinate (B) at (-7.18076:2);
	\coordinate (C) at (30:0.5);
	\coordinate (D) at (-30:0.5);

	\draw [
		decoration={markings, mark=at position 0.15 with {\arrow{>}}},
		postaction={decorate}
		] (A) arc (7.18076:360-7.18076:2);
	\draw [
		decoration={markings, mark=at position 0.4 with {\arrow{>}}},
		postaction={decorate}
		] (C) arc (30:360-30:0.5);
	\draw [
		decoration={markings, mark=at position 0.5 with {\arrow{>}}},
		postaction={decorate}
		] (C) -- (A);
	\draw [
		decoration={markings, mark=at position 0.5 with {\arrow{>}}},
		postaction={decorate}
		] (D) -- (B);

	\node [minimum width=2.3cm, minimum height=0.2cm, inner sep=0pt, pattern=north east lines wide] at (1.15,0) {};

	\node [above right, font=\large] at (2.5,0) {\(x\)};
	\node [above right, font=\large] at (0,2.5) {\(i\,y\)};
	\node [above right] at (59.026532:2) {\(\g_1\)};
	\node [below=0.1cm] at (1.21,-0.25) {\(\g_2\)};
	\node [above left] at (150:0.5) {\(\g_3\)};
	\node [above=0.1cm] at (1.21,0.25) {\(\g_4\)};

	\node [font=\tiny] (E) at (-60:2.5) {\(\ln z\in\R+2\p\,i\)};
	\draw [bend left, ->, help lines] (E.north) to (0.81,-0.1);
	\node [font=\tiny] (F) at (20:2.7) {\(\ln z \in \R\)};
	\draw [bend right, ->, help lines] (F.west) to (1.61,0.1);
}
	\]
	dove
	\[
		\g_1\colon [\q, 2\p-\q] \to \C, t \mapsto R\,e^{i\,t}; \qquad \g_2\colon [\e,R] \to \C, t \mapsto t-i\,\d;
	\]
	e
	\[
		\g_3\colon [\l, 2\p-\l] \to \C, t \mapsto r\,e^{i\,t}; \qquad \g_4\colon [\e,R] \to \C, t \mapsto t+i\,\d.
	\]
	Per il teorema dei residui avremo
	\[
		2\p\,i \sum_i \Res(f,z_i) = \int\limits_{\mathclap{\g_1-\g_2-\g_3+\g_4}}\, f(z)\,\dd z.
	\]
	In particolare
	\[
		\begin{split}
			\int\limits_{\mathclap{\g_1-\g_2-\g_3+\g_4}}\, f(z)\,\dd z & = \int_\q^{2\p-\q} \frac{{(R\,e^{i\,t})}^\a P(R\,e^{i\,t})}{Q(R\,e^{i\,t})}\,i\,R\,e^{i\,t}\,\dd t - \int_\e^R \frac{{(t-i\,\d)}^\a P(t-i\,\d)}{Q(t-i\,\d)}\,\dd t\\
			& - \int_\l^{2\p-\l} \frac{{(r\,e^{i\,t})}^\a P(r\,e^{i\,t})}{Q(r\,e^{i\,t})}\,i\,r\,e^{i\,t}\,\dd t + \int_\e^R \frac{{(t+i\,\d)}^\a P(t+i\,\d)}{Q(t+i\,\d)}\,\dd t.
		\end{split}
	\]
	Osserviamo che
	\[
		\bigg\lvert \frac{P(z)}{Q(z)} \bigg\rvert \le \frac{C}{\abs{z}^2} \text{ se }\abs{z}\ge R_0 \qquad\text{e}\qquad \bigg\lvert \frac{P(z)}{Q(z)} \bigg\rvert \le C' \text{ se }\abs{z}< R_0,
	\]
	da cui
	\[
		\bigg\lvert \int_\q^{2\p-\q} \frac{{(R\,e^{i\,t})}^\a P(R\,e^{i\,t})}{Q(R\,e^{i\,t})}\,i\,R\,e^{i\,t}\,\dd t \bigg\rvert \le \int_\q^{2\p-\q} R^\a \frac{C}{R^2}\,R\,\dd t = R^{\a-1} C\,(2\p-2\q) \xrightarrow{R\to +\infty} 0,\graffito{\(\a<1\)}
	\]
	e
	\[
		\bigg\lvert \int_\l^{2\p-\l} \frac{{(r\,e^{i\,t})}^\a P(r\,e^{i\,t})}{Q(r\,e^{i\,t})}\,i\,r\,e^{i\,t}\,\dd t \bigg\rvert \le \int_\l^{2\p-\l} r^\a C' r\,\dd t = r^{\a+1} C' (2\p-2\l) \xrightarrow{r \to 0} 0\graffito{\(\a>0\)}.
	\]
	Inoltre
	\[
		\int_\e^R \frac{{(t+i\,\d)}^\a P(t+i\,\d)}{Q(t+i\,\d)}\,\dd t \xrightarrow{\d \to 0} \int_\e^R \frac{r^\a P(t)}{Q(t)}\,\dd t \xrightarrow[\e \to 0]{R\to +\infty} \int_0^{+\infty} t^\a \frac{P(t)}{Q(t)}\,\dd t,
	\]
	dove in questo caso tendevamo all'asse reale da sopra, quindi la nostra determinazione del logaritmo ci dice
	\[
		z^\a = e^{\a\,\ln z} \to x^\a.
	\]
	Nell'ultimo caso, quando ciò avviene da sotto l'asse reale, avremo
	\[
		z^\a = e^{\a\,\ln z} = e^{\a\,\ln x}e^{\a\,2\p\,i},
	\]
	da cui
	\[
		\int_\e^R \frac{{(t-i\,\d)}^\a P(t-i\,\d)}{Q(t-i\,\d)}\,\dd t \xrightarrow{\d \to 0} e^{2\p\,i\,\a} \int_\e^R \frac{t^\a P(t)}{Q(t)}\,\dd t \xrightarrow[\e \to 0]{R \to +\infty} e^{2\p\,i\,\a} \int_0^{+\infty} t^\a \frac{P(t)}{Q(t)}\,\dd t.
	\]
	Quindi
	\[
		2\p\,i \sum_i \Res(f,z_i) = \left(1-e^{2\p\,i\,\a}\right) \int_0^{+\infty} t^\a \frac{P(t)}{Q(t)}\,\dd t.
	\]
	A questo punto è sufficiente calcolare i residui di \(f\), i quali corrispondo agli zeri complessi di \(Q\), per completare il calcolo dell'integrale.
\end{sol}